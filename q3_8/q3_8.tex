\documentclass[class=jsarticle, crop=false, dvipdfmx, fleqn]{standalone}
\input{/Users/User/Documents/TeX/preamble/mypreamble}
\begin{document}
\section{練習問題3-8}
連続変数$X$が正規分布$N(\mu, \sigma^2)$に従うとき,
$X$の確率分布は以下の式で表される。
\begin{equation}
p(X) = \frac{1}{\sqrt{2 \pi \sigma^2}} \exp\qty(- \frac{(X - \mu)^2}{2 \sigma^2})
\end{equation}
よって,
\begin{equation}
\begin{split}
H(X) & = E[-\log p(X)] \\
	& = E\qty[- \log\qty{\frac{1}{\sqrt{2 \pi \sigma^2}} \exp\qty(- \frac{(X - \mu)^2}{2 \sigma^2})}] \\
	& = E\qty[\frac{1}{2} \log(2 \pi \sigma^2) + \frac{1}{2 \sigma^2} (X - \mu)^2] \\
	& = \frac{1}{2} \log(2 \pi \sigma^2) + \frac{1}{2 \sigma^2} E[(X - \mu)^2]
\end{split}
\end{equation}
ここで,$E[(X - \mu)^2] = V(X) = \sigma^2$より,
\begin{equation}
\begin{split}
H(X) & = \frac{1}{2} \log(2 \pi \sigma^2) + \frac{1}{2 \sigma^2} \cdot \sigma^2 \\
	& = \frac{1}{2} \qty(\log(2 \pi \sigma^2) + 1) \\
	& = \frac{1}{2} \log(2 \pi e \sigma^2)
\end{split}
\end{equation}


\end{document}